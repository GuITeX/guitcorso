\documentclass[a4paper,12pt]{article}
\usepackage[italian]{babel}
%\usepackage[utf8x]{inputenc}
\usepackage[color]{guit}
\usepackage{hyperref,indentfirst,fancyvrb,cclicenses,pifont,booktabs}

\newcommand{\email}[1]{{\ttfamily <\href{mailto:#1}{#1}>}}

\title{Introduzione al mondo di \LaTeX \\ Guida al corso} 

\author{{\LARGE\GuIT} -- \GuITtext\thanks{Si sono occupati della
realizzazione di questo lavoro: Maurizio W. Himmelmann
\email{himmel@sssup.it}, Emiliano G.~Vavassori
\email{testina@sssup.it} e Fabiano Busdraghi
\email{fabusdr@yahoo.com}.}}
\date{Versione 1.0 --- 31 agosto 2006}

\begin{document}
\maketitle

\begin{abstract}
La presente guida illustra il materiale realizzato per il corso
``Introduzione al mondo di  \LaTeX'' fornendo indicazioni sugli
obiettivi, sullo svolgimento e sull'articolazione delle lezioni. Se ne
consiglia caldamente la lettura prima di utilizzare il materiale.
\end{abstract}

\tableofcontents
\newpage

\section{Note di \textit{copyright}}\label{copy}
Tutto il materiale del corso ``Introduzione al mondo di \LaTeX'' \`e
rilasciato sotto licenza $\!$\cc$\!\!\!$ Creative Commons
2.5\footnote{Il testo completo della licenza \`e disponibile, in
inglese, alla pagina
\url{http://creativecommons.org/licenses/by-nc-sa/2.5/legalcode}.}.

\noindent\emph{Tu sei libero di:}
\begin{dinglist}{227}
    \item di riprodurre, distribuire, comunicare al pubblico, esporre
	in pubblico, rappresentare, eseguire e recitare quest'opera;
    \item di modificare quest'opera.
\end{dinglist}

\noindent\emph{Alle seguenti condizioni:}
\begin{description}
    \item[\ccby Attribuzione] Devi attribuire la paternit\`a
	dell'opera nei modi indicati dall'autore o da chi ti ha dato
	l'opera in licenza.
    \item[\ccnc Non commerciale] Non puoi usare quest'opera per fini
	commerciali.
    \item[\ \ccsa\ \ $\!\!$Condividi allo stesso modo] Se alteri o
	trasformi quest'opera, o se la usi per crearne un'altra, puoi
	distribuire l'opera risultante solo con una licenza identica a
	questa.
\end{description}

\begin{dinglist}{52}
    \item Ogni volta che usi o distribuisci quest'opera, devi farlo
	secondo i termini di questa licenza, che va comunicata con
	chiarezza.
    \item In ogni caso, puoi concordare col titolare dei diritti
	d'autore utilizzi di quest'opera non consentiti da questa
	licenza.
\end{dinglist}

Una ulteriore richiesta che gli autori aggiungono ai termini gi\`a
previsti dalla licenza \`e quella di pubblicizzare l'attivit\`a del
\GuITtext\ utilizzando la classe \textsf{guitbeamer}, appositamente
predisposta da Emiliano G. Vavassori \cite{EVG}.
\section{Obiettivi}
Tra le diverse metodologie disponibili per la promuovere l'uso di
\LaTeX, il suo insegnamento attraverso didattica frontale \`e a nostra
esperienza quello che meglio aiuta il neofita a superare l'impatto
iniziale e ad acquistare la necessaria sicurezza per perseverare
nell'utilizzo. Tuttavia, se la vita del discente risulta enormemente
semplificata dall'essere letteralmente condotto per mano,  altrettanto
non pu\`o dirsi per il docente, il quale oltre alle energie necessarie
per affrontare le fatiche della didattica, deve altres\`i investire
una parte consistente del suo tempo per  la predisposizione del corso
stesso (videoproiezioni, esempi pratici, manuali, ecc.).

Proprio per agevolare chi abbia in animo di organizzare un corso ma
non trovi il tempo e/o la voglia necessaria per allestire tutto il
materiale, abbiamo deciso di condividere l'esperienza maturata in
diversi corsi, svolti secondo schemi gi\`a testati e consolidati,
rendendo pubblico tutto il materiale utilizzato. All'aspirante docente
non rester\`a quindi altro impegno che studiare egli stesso il
materiale fornito, adattandolo eventualmente alle proprie specifiche
esigenze.

\section{Articolazione del corso}

Il corso \`e articolato in quattro lezioni ed \`e rivolto alle persone
senza alcuna conoscenza pregressa di \LaTeX\ e che non possiedano
altres\`i particolare dimestichezza coi linguaggi di programmazione.
Esso pu\`o essere svolto sia per piattaforme Windows che *nix, essendo
stato sperimentato con successo su entrambi i sistemi. Ogni singola
lezione \`e stata rigorosamente tarata per avere una durata
\textit{non superiore} ai 90 minuti (inclusi gli esempi pratici).
Abbiamo resistito alla forte tentazione di aggiungere altro materiale,
anzi talvolta effettuando la (dolorosa) scelta di eliminare alcune
slide per rientrare nei tempi. L'articolazione e la durata delle
lezioni privilegia infatti la fluidit\`a della trattazione
focalizzando gli aspetti di concreto utilizzo e cercando di mantenere
alta la curiosit\`a del discente attraverso l'uso di testi scritti da
autori dalla proverbiale comicit\`a (\cite{BNN,ALL,JKJ}). Nella
possibilit\`a in cui fosse possibile dilatare leggermente i tempi di
ciascuna lezione, \`e possibile utilizzare le slide aggiuntive che nei
sorgenti sono attualmente commentate. A voi l'ardua scelta di
utilizzarle o meno, fermo restando il nostro consiglio di non
aggiungere ulteriore materiale a quello originariamente previsto. 

Per quanto riguarda l'editor la scelta \`e ricaduta su
\textsf{Texmaker}\footnote{Disponibile all'indirizzo internet:
\url{http://www.xm1math.net/texmaker/}} che viene rilasciato sotto
licenza GNU/GPL. Nonostante alcuni limiti (dovuti verosimilmente alla
giovinezza del progetto), ha tuttavia il grande pregio di essere
disponibile per diverse piattaforme (Windows, *nix e MacOS). Questo
ha permesso di ridurre i problemi di piattaforma, minimizzando la
necessit\`a di dovere necessariamente usare uno specifico sistema
operativo. 

\subsection{Prima lezione}
La prima lezione (\texttt{lezione\_1.pdf}) serve essenzialmente per
introdurre la filosofia che sta alla base di \TeX\ e \LaTeX, secondo
un approccio impostato pi\`u sulla strutturazione logico/formale del
documento che sull'editing del testo. Dopo un breve cenno storico, si
chiariscono i concetti fondamentali di \emph{compilazione} e
\emph{codice sorgente}, dando illustrazione della struttura di un
documento minimale. In parallelo alle slide in videoproiezione,
vengono date dimostrazioni pratiche di alcuni editor (generalmente
\textsf{WinEdt}, \textsf{TeXnicCenter} e \textsf{Texmaker} per
piattaforma Windows e \textsf{Kile} e \textsf{Texmaker} per
piattaforma *nix). \`E importante far terminare la lezione con alcuni
esempi di compilazione di documenti minimali per dare da subito l'idea
della semplicit\`a del processo di generazione del documento.
	\begin{itemize}
	\item \TeX\ e \LaTeX
		\begin{itemize}
		\item La storia di \TeX\ 
		\item La compilazione di un documento
		\end{itemize}
	\item Cominciamo a lavorare
		\begin{itemize}
		\item La sintassi dei comandi
		\item La struttura dei sorgenti
		\end{itemize}
	\item Perch\'e scegliere \LaTeX
	\end{itemize}

\subsection{Seconda lezione}
Nel corso della seconda lezione (\texttt{lezione\_2.pdf}) si
forniscono i principali elementi costituenti la struttura del
documento, dando particolare risalto ai vantaggi che quest'approccio
offre sia in termini operativi che di articolazione logica dei
contenuti. Vengono date alcune prime indicazioni sui comandi
fondamentali e sull'impaginazione del testo e vengono illustrati i
riferimenti incrociati ed il loro utilizzo; si passa poi
alla parte conclusiva della lezione in cui si forniscono le regole
tipografiche di base per scrivere documenti non soltanto belli ma
anche tipograficamente congruenti.
	\begin{itemize}
	\item Struttura del documento
		\begin{itemize}
	\item Sezionamento del testo
		\item Elenchi puntati e numerati
		\item Impaginazione con \LaTeX
		\end{itemize}
	\item Riferimenti incrociati
	\item Norme tipografiche di base
		\begin{itemize}
		\item Evidenziare il testo
		\item Sfizi tipografici
		\item ``Dimensionare'' il testo
		\end{itemize}
	\end{itemize}

\subsection{Terza lezione}
La terza lezione (\texttt{lezione\_3.pdf}) costituisce probabilmente
lo ``zoccolo duro'' dell'intero corso e per molti aspetti anche la
parte pi\`u interessante. Vengono infatti esplorate le potenzialit\`a
di \LaTeX\ nella stesura delle tabelle, anche stavolta dopo una breve
introduzione sulle norme minimali in materia di tabelle. La lezione
prosegue poi con le formule matematiche. La parte matematica di
\LaTeX, vista la vastit\`a e l'importanza dell'argomento meriterebbe
da sola un corso ad essa interamente dedicato. Tuttavia, volendo
concentrarci su un pubblico generico e dalle eterogenee esigenze
abbiamo preferito non soffermarci in modo troppo approfondito
lasciando poi, a chi volesse, l'onore/onere di approfondire
ulteriormente l'argomento. Per ultimo, ma non per questo meno
importante, si descrivono le principali metodologie per realizzare la
bibliografia e gli annessi riferimenti. 
	\begin{itemize}
	\item Oggetti flottanti
	\item Tabelle
		\begin{itemize}
		\item Norme tipografiche per le tabelle
		\item Tabulazione
		\item Tabelle
		\item Altri ambienti per le tabelle
		\end{itemize}
	\item Formule matematiche
		\begin{itemize}
		\item Nozioni di base
		\item Scrivere formule matematiche
		\item Ambienti matematici
		\end{itemize}
	\item Riferimenti bibliografici
	\end{itemize}

\subsection{Quarta lezione}
L'ultima lezione (\texttt{lezione\_4.pdf}) si propone di completare
tutti gli argomenti ``standard'' del corso. L'inserimento di figure,
realizzate sia con altri programmi sia con l'ambiente \texttt{picture}
completa il bagaglio minimale di conoscenza per potere cominciare a
gestire autonomamente la creazione di un qualunque documento. Qualche
minuto della lezione viene speso per introdurre il programma
\textsf{Xfig} (o il suo porting su Windows:
\textsf{Winfig}\footnote{Disponibili rispettivamente agli indirizzi
internet: \url{http://www.xfig.org} e
\url{http://www.schmidt-web-berlin.de/WinFIG.htm}}) che permette in
modo facile ed intuitivo di ottenere direttamente il codice da
inserire nel documento. La seconda fase della lezione, dando per
scontato che l'allievo abbia gi\`a acquisito un minimo di esperienza,
illustra due casi di utilizzo pratico di \LaTeX. La classe
\textsf{beamer} per le videoproiezioni\cite{TAN} ed il pacchetto
\textsf{europecv} per curriculum vit\ae\ (che implementa lo standard
della Comunit\`a Europea \cite{VTC}) rappresentano due applicazioni di
concreta utilit\`a per mettere subito in pratica quanto imparato. Il
corso si conclude con qualche suggerimento sull'atteggiamento corretto
per affrontare e superare gli inevitabili problemi e sui principali
canali dove trovare informazioni.
		\begin{itemize}
		\item Figure ed immagini
			\begin{itemize}
			\item Il pacchetto \texttt{graphicx}
			\item Tabulazione
			\item Tabelle
			\item Altri ambienti per le tabelle
			\end{itemize}
		\item Videoproiezioni
			\begin{itemize}
			\item Linee guida per le videoproiezioni
			\item Sintassi di base
			\end{itemize}
		\item Curriculum Vitae
			\begin{itemize}
			\item Sintassi di base
			\end{itemize}
		\item Come sopravvivere a \LaTeX
		\end{itemize}

\subsection{Gli esempi}
Tutte le lezioni sono corredate da 26 esempi, richiamati al momento
opportuno della lezione, che hanno lo scopo di fornire al discente un
immediato esempio pratico di quanto appena visto in forma teorica. Per
rendere compatibile il materiale sia a piattaforma Windows che *nix,
si \`e preferito scrivere il codice in ascii puro. Poich\'e nel
preambolo sono sempre riportate le opzioni di \texttt{inputenc} per
entrambi i sistemi operativi, \`e opportuno che il docente spenda due
parole nella prima lezione su quale delle due opzioni utilizzare. 

All'interno della cartella \texttt{come\_non\_presentare} sono acclusi
anche tre esempi di come assolutamente \textit{non} vadano fatte le
presentazioni a video. Non ce ne vogliano gli autori.

\subsection{Codice sorgente e personalizzazione}
\`E possibile personalizzare a vostro piacimento la presentazione a
condizione di attenersi alle note sul \textit{copyright} (vedi \S
\ref{copy}); in accordo con la filosofia \LaTeX, tutto il codice \`e
stato infatti scritto cercando di agevolare quanto pi\`u possibile la
personalizzazione. Per tutti i dettagli sulla sintassi si rimanda alla
documentazione della classe \textsf{guitbeamer} realizzata da Emiliano
G.~Vavassori appositamente per questo corso \cite{EVG}.

\subsubsection{Pacchetti aggiuntivi e compatibilit\`a}
Di seguito un elenco dei pacchetti che occorre installare per
compilare correttamente le lezioni e gli annessi esempi.

\begin{description}
	\item[\texttt{amscd}:] permette la scrittura di diagrammi
	    commutativi per l'esempio nella terza lezione;
	\item[\texttt{breakurl}:] permette di spezzare gli URL troppo
	    lunghi nelle citazioni bibliografiche;
	\item[\texttt{color}:] permette l'uso di colori all'interno
	    del documento;
	\item[\texttt{colortbl}:] realizza le tabelle colorate nella
	    terza lezione;
	\item[\texttt{eurosym}:] realizza il simbolo dell'euro nella
	    terza lezione;
	\item[\texttt{lipsum}:] genera testo riempitivo, utilizzato in
	    quasi tutti gli esempi;
	\item[\texttt{longtable}:] realizza tabelle su pi\`u pagine
	\item[\texttt{setspace}:] permette di personalizzare la
	    spaziatura tra le righe di un documento per gli esempi
	    della seconda lezione;
	\item[\texttt{soul}:] permette di effettuare diversi tipi di
	    sottolineature, usata opzionalmente negli esempi della
	    seconda lezione; 
	\item[\texttt{textcomp}:] supporto per l'encoding TS1
	    necessari per la seconda lezione;
	\item[\texttt{rotating}:] usato per ruotare oggetti nella
	    terza e quarta lezione;
	\item[\texttt{xy}:] realizza l'esempio nella terza lezione;
\end{description}

\subsection{Sviluppi}
Ovviamente \`e pura utopia pensare di avere messo la parola fine ad un
lavoro come questo. Ogni anno quando si pensava di avere gi\`a
raggiunto un discreto livello di completezza del materiale,  sono
sempre emersi nuovi argomenti meritori di essere integrati nel corso.
Riteniamo pertanto utilissimi tutti i commenti e le proposte di
miglioramento che vorrete farci pervenire\footnote{Scrivete a \GuIT\
--- \GuITtext\ \email{guit@sssup.it}.} in modo da essere in grado di
offrire un prodotto sempre valido ed aggiornato.

\section{Ringraziamenti}
Questo lavoro non sarebbe stato realizzato nella forma a voi visibile
senza la speciale opera di Emanuele Vicentini che si \`e prestato con
infinita pazienza e disponibilit\`a a soddisfare le nostre mille
richieste di aiuto e di \textit{bug solving}. Emanuele, per quello che
hai fatto troverai un giorno $100+70$ immacolati ringraziamenti ad
attenderti \texttt{;-)}

\begin{thebibliography}{666}
\bibitem{ALL}
	Allen, Woody.
	\newblock \textit{Saperla lunga}.
	\newblock Bompiani
\bibitem{BNN}
	Benni, Stefano.
	\newblock \textit{Il dottor Ni\`u}.
	\newblock Feltrinelli
\bibitem{CAS}
	Caschili, Massimo.
	\newblock \textit{Semplici Figure con l'Ambiente Picture}.
	\newblock  \Ars, 1/2006
\bibitem{CEV}
	Cevolani, Gustavo.
	\newblock \textit{Norme Tipografiche per l'Italiano in \LaTeX}.
	\newblock \Ars, 1/2006
\bibitem{FER}
	Fear, Simon.
	\newblock \textit{Publication quality tables in \LaTeX}.
	\newblock \url{http://www.ctan.org/tex-archive/macros/latex/contrib/booktabs/booktabs.pdf}
\bibitem{GIN}
	Gini, Rosa.
	\newblock\textit{Corso \LaTeX, materiale del corso di R. Gini destinato al personale dell'Agenzia Regionale di Sanit\`a della Toscana}.
	\newblock \url{http://www.guit.sssup.it/corsi/corso\_rosa.php/}
\bibitem{JKJ}
	Jerome K., Jerome.
	\newblock \textit{Tre uomini in barca}.
	\newblock Mondadori
\bibitem{SYR}
	Syropoulos, Apostolos; Tsolomitis, Antonis; Sofroniou, Nick.
	\newblock \textit{Digital Typography using \LaTeX}.
\bibitem{TAN}
	Tantau, Till.
	\newblock\textit{User's Guide to the Beamer Class}.
	\newblock \url{http://latex-beamer.sourceforce.net/}
\bibitem{EVG}
	Vavassori, Emiliano.
	\newblock\textit{La classe per presentazioni \textsf{guitbeamer}}.
	\newblock
	\url{http://tug.ctan.org/tex-archive/macros/latex/contrib/GuIT/guitbeamer/}
\bibitem{VTC}
	Vitacolonna, Nicola.
	\newblock\textit{europecv an Unofficial Class for European Curricula}.
	\newblock \url{http://www.ctan.org/tex-archive/macros/latex/contrib/europecv/}
\end{thebibliography}


\end{document}
