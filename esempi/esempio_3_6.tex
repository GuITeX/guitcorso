\documentclass[a4paper,12pt]{article}

\usepackage[italian]{babel}
\usepackage[utf8x]{inputenc}
%\usepackage[latin1]{inputenc}


\usepackage{amsmath} % carica funzionalità matematiche aggiuntive 


\begin{document}


% esempio senza \text{}

Lorem ipsum dolor sit amet, consectetuer adipiscing elit. Ut purus elit, vestibulum ut, placerat ac, 
\begin{displaymath}
	\forall x \in A noi abbiamo che x^2 =1  
\end{displaymath}
adipiscing vitae, felis. Curabitur dictum gravida mauris.




% % cambiare stile al testo
% 
% Lorem ipsum dolor sit amet, consectetuer adipiscing elit. Ut purus elit, vestibulum ut, placerat ac, 
% \begin{displaymath}
% 	\forall x \in A \textbf{ noi abbiamo che } x^2 =1  
% \end{displaymath}
% adipiscing vitae, felis. Curabitur dictum gravida mauris.





% % cambiare stile al testo
% 
% Lorem ipsum dolor sit amet, consectetuer adipiscing elit. Ut purus elit, vestibulum ut, placerat ac, 
% \begin{displaymath}
% 	\forall x \in \mathcal{A} \text{ noi abbiamo che } x^2 =1  
% \end{displaymath}
% adipiscing vitae, felis. Curabitur dictum gravida mauris.





% % cambiare stile al testo
% 
% Lorem ipsum dolor sit amet, consectetuer adipiscing elit. Ut purus elit, vestibulum ut, placerat ac, 
% \begin{displaymath}
% 	\forall \mathsf{x} \in \mathcal{A} \text{ noi abbiamo che } x^2 =1  
% \end{displaymath}
% adipiscing vitae, felis. Curabitur dictum gravida mauris.

\end{document}
