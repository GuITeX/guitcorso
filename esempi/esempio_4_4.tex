\documentclass[a4paper,10pt]{article}

\usepackage[italian]{babel}
\usepackage[utf8x]{inputenc}
%\usepackage[latin1]{inputenc}
\usepackage{lipsum} 

\usepackage[pdftex]{graphicx} % gestisce immagini e figure
\usepackage{color} % permette di utilizzare i colori (necessario per Xfig) 


\begin{document}


\lipsum[1]
\setlength{\unitlength}{3947sp}%
%
\begingroup\makeatletter\ifx\SetFigFont\undefined%
\gdef\SetFigFont#1#2#3#4#5{%
  \reset@font\fontsize{#1}{#2pt}%
  \fontfamily{#3}\fontseries{#4}\fontshape{#5}%
  \selectfont}%
\fi\endgroup%
\begin{picture}(3400,2949)(663,-2998)
{\color[rgb]{0,0,0}\thinlines
\put(1276,-811){\circle{1210}}
}%
{\color[rgb]{0,0,0}\put(676,-2986){\framebox(3375,2925){}}
}%
\end{picture}%





\lipsum[2]


\end{document}
