\documentclass[12pt]{beamer}

\usepackage[italian]{babel}
\usepackage[utf8x]{inputenc}
%\usepackage[latin1]{inputenc}

\usetheme{Madrid}
% \usecolortheme{crane}

\title{Il quinto teorema di Vr\o dstat} % titolo, autore, data  
\author{Vito Catozzo}
\date{\today}

\begin{document}


\frame{\titlepage} % pagina del titolo
%-----------------------------------------------------------
%--------------------------------------------------- SLIDE -
\begin{frame}
  \frametitle{Piano della presentazione}
  \tableofcontents
\end{frame}
%-----------------------------------------------------------
%--------------------------------------------------- SLIDE -
\begin{frame}
  \frametitle{Guide gratuite}
	\begin{thebibliography}{Wel66}
		\bibitem [LL94]{CATOZZO}
			Catozzo, Vito.
			\newblock \textit{La Pallottola Vagante}.
		\bibitem [LM05]{MORI}
			Musumeci, Calogero.
			\newblock \textit{La trebbiatura dei ficodindia}.
	\end{thebibliography}
\end{frame}
%-----------------------------------------------------------
%--------------------------------------------------- SLIDE -
\begin{frame}
  \frametitle{Cosa mangiamo?}
	\begin{columns}
	  \column[t]{.5\textwidth}
		Kitekat alla magrebina
	  \column[t]{.5\textwidth}
		Bocconcini barboncini alla Pompadour
	\end{columns}
\end{frame}
%-----------------------------------------------------------
%--------------------------------------------------- SLIDE -
\begin{frame}
  \frametitle{Menù di stasera}
	\begin{block}{Plaisir du chef}
		Polpettone di bocconcini per gatti, ingentilito da una salsa agrodolce a base di sugo di locuste e Pernod. Piatto vigoroso ma ricco di sfumature. 
	\end{block}
\end{frame}
%-----------------------------------------------------------
%--------------------------------------------------- SLIDE -
\end{document}
