\documentclass[a4paper,12pt]{article}

\usepackage[italian]{babel}
\usepackage[utf8x]{inputenc}
%\usepackage[latin1]{inputenc}

\usepackage{amsmath}


\begin{document}

% esempio di equation

Lorem ipsum dolor sit amet, consectetuer adipiscing elit. Ut purus elit, vestibulum ut, placerat ac, 
\begin{displaymath} 
	\begin{array}{l}
		x+y+z=0\\
		2x-y=1\\
		y-4z=-3 
	\end{array} 
\end{displaymath} 
adipiscing vitae, felis. Curabitur dictum gravida mauris.
\begin{equation}
	\begin{array}{l}
		x+y+z=0\\
		2x-y=1\\
		y-4z=-3 
	\end{array} 
\end{equation}
Nam arcu libero, nonummy eget, consectetuer id, vulputate a, magna.
\begin{equation} 
\left\{
	\begin{array}{l}
	x+y+z=0\\
	2x-y=1\\
	y-4z=-3 
	\end{array} 
\right. 
\end{equation} 
Donec vehicula augue eu neque.


% % esempio di cases e multiline
 
 Lorem ipsum dolor sit amet, consectetuer adipiscing elit. Ut purus elit, vestibulum ut, placerat ac, 
 \begin{displaymath}
 f(n):=
 	\begin{cases} 
 	2n+1 & \text{se $n$ è dispari,}\\ 
 	n/2  & \text{se $n$ è pari.} \\
 	\end{cases}
 \end{displaymath}
 adipiscing vitae, felis. Curabitur dictum gravida mauris.
 \begin{multline} 
 	f=a+b+c+d+e+g+h \\ 
 	+i+k+l+m+n+o+\\ 
 	+p+q+r+s+t+u+v 
 \end{multline} 
 Nam arcu libero, nonummy eget, consectetuer id, vulputate a, magna.



\end{document}
