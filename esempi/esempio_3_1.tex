\documentclass[a4paper,10pt]{article}

\usepackage[italian]{babel}
\usepackage[utf8x]{inputenc}
%\usepackage[latin1]{inputenc}
\usepackage{lipsum} 

% \usepackage{float} % permette di forzare il posizionamento dei float


\begin{document}


% \lipsum[1]


\begin{tabular}{l r r}
Votantonio	&	12 &	45,3 \\
Votantonio	&	15 &	78,5 \\
Votantonio	&	17 & 	34,0 \\
\end{tabular}


% \lipsum[1]


 \begin{table}[htb] % \caption{Risultati delle elezioni}
 \begin{tabular}{l r r}
 Votantonio	&	12 &	45,3 \\
 Votantonio	&	15 &	78,5 \\
 Votantonio	&	17 & 	34,0 \\
 \end{tabular}
 \end{table}


% \lipsum[1]



 \begin{table}[htb] \caption{Risultati delle elezioni}
   \centering
 	\begin{tabular}{p{.3\textwidth} r r}\hline
 		\multicolumn{1}{l}{Candidati}
 		&\multicolumn{1}{c}{Votanti}
 		&\multicolumn{1}{c}{\%}\\\hline
 		Antonio La Trippa	&	1.200 &	27,3 \\
 		Antonio La Trippa	&	1.500 &	34,1 \\
 		Antonio La Trippa	&	1.700 & 38,6 \\\hline
 	\end{tabular}
 \end{table}


\end{document}
