\begin{frame}
  \frametitle{Guide consigliate}
	\begin{thebibliography}{Fer66}
		\bibitem [LC05]{CAU}
			Caucci, Luca; Spadaccini, Mariano.
			\newblock\textit{Gestione di Figure e Tabelle con \LaTeX}.
			\newblock{\small\url{http://www.guit.sssup.it/download/fig_tut.pdf}} 
		\bibitem [SF05]{FEA}
			Fear, Simon.
			\newblock\textit{Publication quality tables in \LaTeX}.
			\newblock{\small\burl{http://www.ctan.org/tex-archive/macros/latex/contrib/booktabs/booktabs.pdf}}
		\bibitem [AA99]{AMS} 
			American Mathematical Society.
			\newblock\textit{User's Guide for the AMSmath Package}
			\newblock{\small\url{ftp://ftp.ams.org/pub/tex/doc/amsmath/amsldoc.pdf}}
	\end{thebibliography}
\end{frame}
%-----------------------------------------------------------
%--------------------------------------------------- SLIDE -
\begin{frame}
  \frametitle{Piano della presentazione}
  \tableofcontents
\end{frame}
%-----------------------------------------------------------
%------------------------------------------------- SECTION -
\section{Oggetti flottanti}
%-----------------------------------------------------------
%--------------------------------------------------- SLIDE -
\begin{frame}
  \frametitle{Oggetti flottanti}
	In \LaTeX\ esiste la possibilit\`a di inserire oggetti (figure o tabelle) esattamente nel punto in cui essi sono posizionati nel testo.\\
  \smallskip
	Tuttavia questo \`e \textbf{da evitare}, perch\'e una delle peculiarit\`a di \LaTeX\ \`e la capacit\`a di inserire oggetti nel posto giudicato ottimale in base a precisi canoni tipografici.\\
  \bigskip
	Per questo tabelle e figure sono detti \textbf{oggetti mobili o flottanti} (\emph{floating}).
\end{frame}
%-----------------------------------------------------------
%--------------------------------------------------- SLIDE -
\begin{frame}
  \frametitle{Oggetti flottanti}
	Posizionata all'interno di un oggetto flottante l'opzione:
	\begin{LaTeXcode}
		\tls htb\trs 
	\end{LaTeXcode}	
	Esprime la nostra preferenza circa la posizione nella pagina ove l'oggetto debba essere posizionato. 
  \medskip
	\begin{itemize}
		\item \Lopt{h} posizionalo `qui' (\emph{here})
		\item \Lopt{t} oppure posizionalo `in cima' (\emph{top})
		\item \Lopt{b} o ancora possibile posizionalo `in fondo' (\emph{bottom}); 
		\item \Lopt{p} o eventualmente posizionalo su una pagina dedicata a tutti gli oggetti \textit{float}; 
	\end{itemize}
	Naturalmente l'ordine \`e modificabile
\end{frame}
%-----------------------------------------------------------
%--------------------------------------------------- SLIDE -
\begin{frame}
  \frametitle{Oggetti flottanti}
	Se si desidera posizionare l'oggetto nel punto esatto in cui si trova (da evitare assolutamente!) si pu\`o aggiungere un punto esclamativo
	\begin{LaTeXcode}
		\tls h!\trs 
	\end{LaTeXcode}	
  \medskip
  \onslide<2->
	Alternativamente si pu\`o usare il pacchetto \LCmd[]{float}
	\begin{LaTeXcode}
		\\usepackage\{\alert{float}\}
	\end{LaTeXcode}	
	\begin{LaTeXcode}
		\tls H\trs
	\end{LaTeXcode}	
\end{frame}
%-----------------------------------------------------------
%--------------------------------------------------- SLIDE -
\begin{frame}
  \frametitle{Raccomandazioni sul posizionamento}
	\begin{center}
		\textbf{Fidatevi di \LaTeX!}\\
		Se fisicamente non c'\`e spazio inutile insistere.
	\end{center}
  \medskip
	\begin{block}{Posizionamento ottimale}
		\begin{itemize}
			\item \Lopt{tb} per oggetti `normali'
			\item \Lopt{p} per oggetti `grandi'
		\end{itemize}
	\end{block}
	Fino ad aver completato il documento non preoccupatevi minimamente del posizionamento. In fase di revisione potete usare il pacchetto \Lsty{placeins}.
\end{frame}
%-----------------------------------------------------------
%------------------------------------------------- SECTION -
\section{Tabelle}
%-----------------------------------------------------------
%--------------------------------------------------- SLIDE -
\begin{frame}
  \frametitle{A che punto siamo}
  \tableofcontents[currentsection,currentsubsection]
\end{frame}
%-----------------------------------------------------------
%---------------------------------------------- SUBSECTION -
\subsection{Norme tipografiche per le tabelle}
%-----------------------------------------------------------
%--------------------------------------------------- SLIDE -
\begin{frame}
  \frametitle{Il \textit{layout} di una tabella}
	La tabella \`e un oggetto che \`e stato definito nel corso di secoli di esperienza e che dovrebbe essere alterato solo in circostanze eccezionali. Purtroppo talvolta capita di imbattersi in \textit{tableau} di questo tipo: 
	\begin{LaTeXoutput}\centering
		\begin{tabular}{||r|ll||} \hline
		\multicolumn{1}{||c|}{topi}	& in salm\`i	& \euro{}13,65 \\ \cline{2-3}
			& crudi		& ,50 \\ \hline
		alce	& stufata	& 92,50 \\ \cline{1-1} \cline{3-3}
		bradipi	&		& 33,333 \\ \hline
		armadillo & congelato	& \multicolumn{1}{r||}{8,99} \\ \hline
		\end{tabular}
	\end{LaTeXoutput}
\end{frame}
%-----------------------------------------------------------
%--------------------------------------------------- SLIDE -
\begin{frame}
  \frametitle{La struttura di una tabella formale}
	La tabella risulta molto pi\`u chiara se si utilizzano solo righe orizzontali
	\begin{LaTeXoutput}\centering
		\begin{tabular}{@{}llr@{}} \hline
		\multicolumn{2}{c}{Item} &  Prezzo (\euro{})\\\cline{1-2}
		Animale & Descrizione	& \\ \hline
		Topi	& in salm\`i	& 13,65 \\ 
			& crudi		& 0,50 \\ 
		Alce	& stufata	& 92,50 \\ 
		Bradipi	&		& 33,33 \\ 
		Armadillo & congelato	& 8,99 \\ \hline
		\end{tabular}
	\end{LaTeXoutput}
\end{frame}
%-----------------------------------------------------------
%--------------------------------------------------- SLIDE -
\begin{frame}
  \frametitle{La struttura di una tabella formale}
	Bastano poche regole per essere sicuri di non commettere errori:
  \medskip
	\begin{itemize}
		\item \textbf{mai} usare \textbf{righe verticali} (meno che mai	doppie righe)
		\item specificare sempre l'unit\`a  di misura nell'\textbf{intestazione} di colonna (mai nel corpo della tabella)
		\item \textbf{allineare} i numeri a destra ed il testo a sinistra 
		\item usare sempre lo stesso numero di decimali ed, ove occorra, farli precedere da uno zero (0,15 e \emph{non} ,15)
		\item \textbf{mai} usare \textbf{virgolette} o segni di dubbia natura per ripetere un valore precedente (o riscrivere il valore o lasciare la cella bianca)
	\end{itemize}
\end{frame}
%-----------------------------------------------------------
%---------------------------------------------- SUBSECTION -
\subsection{Tabulazione}
%-----------------------------------------------------------
%--------------------------------------------------- SLIDE -
\begin{frame}
  \frametitle{A che punto siamo}
  \tableofcontents[currentsection,currentsubsection]
\end{frame}
%-----------------------------------------------------------
%--------------------------------------------------- SLIDE -
\begin{frame}
  \frametitle{La tabulazione}
	\begin{LaTeXcode}
		\\begin\{tabular\}\{\alert{lr}\}\n
		\hspace*{5ex}bianco \alert{\&} 102,5 \alert{\bs\bs}\n
		\hspace*{5ex}nero   \alert{\&} 15,4 \alert{\bs\bs}\n
		\\end\{tabular\}
	\end{LaTeXcode}
	\begin{LaTeXoutput}
		\begin{tabular}{l r}
		bianco & 102,5 \\
		nero   & 15,4 \\
		\end{tabular}
	\end{LaTeXoutput}
\end{frame}
-----------------------------------------------------------
--------------------------------------------------- SLIDE -
\begin{frame}
  \frametitle{La tabulazione}
	\begin{LaTeXcode}
		\\begin\{tabular\}\{lr\}\
		\alert{\\hline}\n
		\hspace*{5ex}bianco \& 102,5 \bs\bs\n
		\hspace*{5ex}nero   \& 15,4 \bs\bs\
		\alert{\\hline}\n
		\\end\{tabular\}
	\end{LaTeXcode}
	\begin{LaTeXoutput}
		\begin{tabular}{l r}
		\hline
		bianco & 102,5 \\
		nero   & 15,4 \\
		\hline
		\end{tabular}
	\end{LaTeXoutput}
\end{frame}
%-----------------------------------------------------------
%--------------------------------------------------- SLIDE -
\begin{frame}
  \frametitle{La tabulazione}
	\begin{LaTeXcode}
		\\begin\{tabular\}\{lr\}\
		\\hline \n
		\alert{\\multicolumn\{2\}\{c\}\{intestazione\}\bs\bs} \
		\\hline \n
		\hspace*{5ex}bianco \& 102,5 \bs\bs\n
		\hspace*{5ex}nero   \& 15,4 \bs\bs\
		\\hline\n
		\\end\{tabular\}
	\end{LaTeXcode}
	\begin{LaTeXoutput}
		\begin{tabular}{lr}\hline
		\multicolumn{2}{c}{intestazione}\\\hline
		bianco & 102,5 \\
		nero   & 15,4 \\\hline
		\end{tabular}
	\end{LaTeXoutput}
\end{frame}
%-----------------------------------------------------------
%--------------------------------------------------- SLIDE -
\begin{frame}
  \frametitle{La tabulazione}
	\begin{LaTeXcode}
		\\begin\{tabular\}\{\alert{p\{3cm\}}r\}\
		\\hline \n
		\\multicolumn\{2\}\{c\}\{intestazione\}\bs\bs \
		\\hline \n
		\hspace*{5ex}bianco \& 102,5 \bs\bs\n
		\hspace*{5ex}nero   \& 15,4 \bs\bs\
		\\hline\n
		\\end\{tabular\}
	\end{LaTeXcode}
	\begin{LaTeXoutput}
		\begin{tabular}{p{3cm}r}\hline
		\multicolumn{2}{c}{intestazione}\\\hline
		bianco & 102,5 \\
		nero   & 15,4 \\\hline
		\end{tabular}
	\end{LaTeXoutput}
\end{frame}
%-----------------------------------------------------------
%--------------------------------------------------- SLIDE -
\begin{frame}
  \frametitle{La tabulazione}
	\begin{LaTeXcode}
		\\begin\{tabular\}\{\alert{p\{.3\\textwidth\}}r\}\
		\\hline \n
		\\multicolumn\{2\}\{c\}\{intestazione\}\bs\bs \
		\\hline \n
		\hspace*{5ex}bianco \& 102,5 \bs\bs\n
		\hspace*{5ex}nero   \& 15,4 \bs\bs\
		\\hline\n
		\\end\{tabular\}
	\end{LaTeXcode}
	\begin{LaTeXoutput}
		\begin{tabular}{p{.3\textwidth}r}\hline
		\multicolumn{2}{c}{intestazione}\\\hline
		bianco & 102,5 \\
		nero   & 15,4 \\\hline
		\end{tabular}
	\end{LaTeXoutput}
\end{frame}
%-----------------------------------------------------------
%--------------------------------------------------- SLIDE -
\begin{frame}
  \frametitle{Uso del \textit{pipe}}
	Se proprio non potete evitare di inserire le righe verticali, usate il comando \LCmd[]{|} (\textit{pipe})
	\begin{LaTeXcode}
		\\begin\{tabular\}\{\alert{|}p\{.3\\textwidth\}\alert{|}r\alert{|}\}\
		\\hline \n
		\\multicolumn\{2\}\{\alert{|}c\alert{|}\}\{intestazione\}\bs\bs \
		\\hline \n
		\hspace*{5ex}bianco \& 102,5 \bs\bs\n
		\hspace*{5ex}nero   \& 15,4 \bs\bs\
		\\hline\n
		\\end\{tabular\}
	\end{LaTeXcode}
	\begin{LaTeXoutput}
		\begin{tabular}{|p{.3\textwidth}|r|}\hline
		\multicolumn{2}{|c|}{intestazione}\\\hline
		bianco & 102,5 \\
		nero   & 15,4 \\\hline
		\end{tabular}
	\end{LaTeXoutput}
\end{frame}
%-----------------------------------------------------------
%--------------------------------------------------- SLIDE -
\begin{frame}
  \frametitle{La tabulazione}
	\begin{LaTeXcode}
		\alert{\\begin\{center\}}\n
		\\begin\{tabular\}\{p\{.3\\textwidth\}r|\}\
		\\hline \n
		\\multicolumn\{2\}\{c\}\{intestazione\}\bs\bs \
		\\hline \n
		\hspace*{5ex}bianco \& 102,5 \bs\bs\n
		\hspace*{5ex}nero   \& 15,4 \bs\bs\
		\\hline\n
		\\end\{tabular\}\n
		\alert{\\end\{center\}}
	\end{LaTeXcode}
	\begin{LaTeXoutput}
	  \begin{center}
		\begin{tabular}{p{.3\textwidth}r}\hline
		\multicolumn{2}{c}{intestazione}\\\hline
		bianco & 102,5 \\
		nero   & 15,4 \\\hline
		\end{tabular}
	  \end{center}
	\end{LaTeXoutput}
\end{frame}

%-----------------------------------------------------------
%--------------------------------------------------- SLIDE -
\begin{frame}
  \frametitle{La tabulazione}
	\begin{LaTeXcode}
		\alert{\{\\centering}\n
		\\begin\{tabular\}\{p\{.3\\textwidth\}r\}\
		\\hline \n
		\\multicolumn\{2\}\{c\}\{intestazione\}\bs\bs \
		\\hline \n
		\hspace*{5ex}bianco \& 102,5 \bs\bs\n
		\hspace*{5ex}nero   \& 15,4 \bs\bs\
		\\hline\n
		\\end\{tabular\}\n
		\alert{\}}
	\end{LaTeXcode}
	\begin{LaTeXoutput}\centering
		\begin{tabular}{p{.3\textwidth}r}\hline
		\multicolumn{2}{c}{intestazione}\\\hline
		bianco & 102,5 \\
		nero   & 15,4 \\\hline
		\end{tabular}
	\end{LaTeXoutput}
\end{frame}
%-----------------------------------------------------------
%--------------------------------------------------- SLIDE -
\begin{frame}
  \frametitle{Colorare le righe}
	\begin{LaTeXcode}
		\\begin\{tabular\}\{p\{.3\\textwidth\}r\}\
		\\hline \n
		\hspace*{5ex}\alert{\\rowcolor[gray]\{.7\}} bianco \& 102,5 \bs\bs\n
		\hspace*{5ex}\alert{\\rowcolor[gray]\{.8\}} nero   \& 15,4 \bs\bs\
		\\hline\n
		\\end\{tabular\}
	\end{LaTeXcode}
	\begin{LaTeXoutput}\centering
		\begin{tabular}{p{.3\textwidth}r}\hline
		\rowcolor[gray]{.7} bianco & 102,5 \\
		\rowcolor[gray]{.8} nero   & 15,4 \\\hline
		\end{tabular}
	\end{LaTeXoutput}
  \onslide<2->
	\begin{block}{Attenzione!}
		\LCmd{rowcolor} richiede il pacchetto \LCmd[]{colortbl}
	\end{block}
\end{frame}
%-----------------------------------------------------------
%---------------------------------------------- SUBSECTION -
\subsection{Tabelle}
%-----------------------------------------------------------
%--------------------------------------------------- SLIDE -
\begin{frame}
  \frametitle{La tabella}
	\begin{LaTeXcode}
		\alert{\\begin\{table\}}\n
		\\begin\{tabular\}\{p\{.3\\textwidth\}r\}\
		\\hline \n
		\\multicolumn\{2\}\{c\}\{intestazione\}\bs\bs \
		\\hline \n
		\hspace*{5ex}bianco \& 102,5 \bs\bs\n
		\hspace*{5ex}nero   \& 15,4 \bs\bs\
		\\hline\n
		\\end\{tabular\}\n
		\alert{\\end\{table\}}
	\end{LaTeXcode}
	\begin{LaTeXoutput}\centering
% 		\begin{table} % nel block l'ambiente table aggiunge un rigo a pi\'e di tabella :|
		\begin{tabular}{p{.3\textwidth}r}\hline
		\multicolumn{2}{c}{intestazione}\\\hline
		bianco & 102,5 \\
		nero   & 15,4 \\\hline
		\end{tabular}
% 		\end{table}
	\end{LaTeXoutput}
\end{frame}
%-----------------------------------------------------------
%--------------------------------------------------- SLIDE -
\begin{frame}
  \frametitle{La tabella}
	\begin{LaTeXcode}
		\\begin\{table\}\alert{\tls htb\trs}\n
		\\begin\{tabular\}\{p\{.3\\textwidth\}r\}\
		\\hline \n
		\\multicolumn\{2\}\{c\}\{intestazione\}\bs\bs \
		\\hline \n
		\hspace*{5ex}bianco \& 102,5 \bs\bs\n
		\hspace*{5ex}nero   \& 15,4 \bs\bs\
		\\hline\n
		\\end\{tabular\}\n
		\\end\{table\}
	\end{LaTeXcode}
	\begin{LaTeXoutput}\centering
% 		\begin{table} % nel block l'ambiente table aggiunge un rigo a pi\'e di tabella :|
		\begin{tabular}{p{.3\textwidth}r}\hline
		\multicolumn{2}{c}{intestazione}\\\hline
		bianco & 102,5 \\
		nero   & 15,4 \\\hline
		\end{tabular}
% 		\end{table}
	\end{LaTeXoutput}
\end{frame}
%-----------------------------------------------------------
%--------------------------------------------------- SLIDE -
\begin{frame}
  \frametitle{La didascalia}
	La didascalia si inserisce con il comando \LCmd{caption}, che come i comandi di sezionamento offre la possibilit\`a di specificare un titolo breve per l'indice delle figure.
	\begin{LaTeXcode}
		\alert{\\caption\{}Titolo della tabella\alert{\}}
	\end{LaTeXcode}
  \medskip
 	 Per citare l'oggetto flottante \`e sufficente inserire una \LCmd{label} \textbf{dopo} la \LCmd{caption}\\
\end{frame}
%-----------------------------------------------------------
%--------------------------------------------------- SLIDE -
\begin{frame}
  \frametitle{La didascalia della tabella}
	\begin{LaTeXcode}
		\\begin\{table\}[htb]\alert{\\caption\{titolo\}\\label\{ciccio\}}\n
		\\begin\{tabular\}\{p\{.3\\textwidth\}r\}\
		\\hline \n
		\\multicolumn\{2\}\{c\}\{intestazione\}\bs\bs \
		\\hline \n
		\hspace*{5ex}bianco \& 102,5 \bs\bs\n
		\hspace*{5ex}nero   \& 15,4 \bs\bs\
		\\hline\n
		\\end\{tabular\}\n
		\\end\{table\}
	\end{LaTeXcode}
\end{frame}
%-----------------------------------------------------------
%--------------------------------------------------- SLIDE -
\begin{frame}
  \frametitle{La didascalia della tabella}
	Ecco come appare la tabella 1
	\begin{LaTeXoutput}
	  \begin{center}
		\begin{table}
		\begin{tabular}{p{.3\textwidth}r}
		\multicolumn{2}{c}{Tabella 1: titolo}\\[1ex]\hline %block non digerisce \caption
		\multicolumn{2}{c}{intestazione}\\\hline
		bianco & 102,5 \\
		nero   & 15,4 \\\hline
		\end{tabular}
		\end{table}
	  \end{center}
	\end{LaTeXoutput}
  \bigskip
	\begin{block}{Attenzione!}
		Per modificare lo stile delle didascalie si usa il pacchetto \Lsty{caption}
	\end{block}
\end{frame}
%-----------------------------------------------------------
%--------------------------------------------------- SLIDE -
\begin{frame}
  \frametitle{Un esempio vale pi\`u di mille parole}
	\begin{center}
		\alert{\texttt{esempio\_3\_1.tex}}
	\end{center}
\end{frame}
%-----------------------------------------------------------
%---------------------------------------------- SUBSECTION -
\subsection{Altri ambienti per tabelle}
%-----------------------------------------------------------
%--------------------------------------------------- SLIDE -
\begin{frame}
  \frametitle{A che punto siamo}
  \tableofcontents[currentsection,currentsubsection]
\end{frame}
%-----------------------------------------------------------
%--------------------------------------------------- SLIDE -
\begin{frame}
  \frametitle{Altri ambienti per tabelle}
	Oltre all'ambiente \Lsty{table} esistono anche altri pacchetti per realizzare tabelle. Questi i pi\`u comuni:
	\begin{itemize}
		\item \Lsty{longtable}: tabelle che proseguono nella pagina successiva
		\item \Lsty{sideways}: tabelle ruotate di 90$^{\circ}$ sulla pagina
	\end{itemize}
	I pacchetti specifici devono essere richiamati nel preambolo.
\end{frame}
%-----------------------------------------------------------
%--------------------------------------------------- SLIDE -
\begin{frame}
  \frametitle{Tabelle su pi\`u pagine}
	Talvolta la tabella \`e cos\`i lunga che deve continuare nella pagina successiva
	\begin{LaTeXcode}
		\alert{\\begin\{longtable\}\{lr\}}\n
		\\caption\{titolo\}\\label\{a\}\
		\\hline\n
		\hspace*{5ex}bianco \& 102,5 \bs\bs\n
		\hspace*{5ex}nero   \& 15,4 \bs\bs\
		\\hline\n
		\alert{\\end\{longtable\}}
	\end{LaTeXcode}
  \onslide<2->
	\begin{block}{Attenzione!}
		\`E necessario caricare il pacchetto \Lsty{longtable}
	\end{block}
\end{frame}
%-----------------------------------------------------------
%--------------------------------------------------- SLIDE -
\begin{frame}
  \frametitle{Tabelle su pi\`u pagine}
	\begin{LaTeXoutput}
		\begin{longtable}{lr}
		\multicolumn{2}{c}{Tabella 1: titolo}\\[1ex]\hline %block non digerisce \caption
		bianco		& 10,2 \\
		nero		& 15,6 \\
		giallo		& 16,6 \\
		fucsia		& 15,7 \\
		cremisi		& 12,2 \\
		amaranto	& 18,3 \\
		verde		& 11,5 \\
		grigio		& 15,3 \\
		viola		& 19,9 \\
		blu		& 14,7 \\
		azzurro		& 16,3 
		\end{longtable}
	\end{LaTeXoutput}
\end{frame}
%-----------------------------------------------------------
%--------------------------------------------------- SLIDE -
\begin{frame}
  \frametitle{Tabelle su pi\`u pagine}
	\begin{LaTeXoutput}
		\begin{longtable}{lr} % andr\`o all'inferno per questo :\ ma block non capisce longtable
		rosso		& 14,4 \\
		marrone		& 17,7 \\
		rosa		& 12,9 \\
		ocra		& 19,2 \\
		arancione	& 11,8 \\
		porpora		& 14,6 \\
		celeste		& 12,9 \\
		antracite	& 15,1 \\
		\hline
		\end{longtable}
	\end{LaTeXoutput}
\end{frame}
%-----------------------------------------------------------
%--------------------------------------------------- SLIDE -
\begin{frame}
  \frametitle{Ruotare le tabelle}
	\begin{LaTeXcode}
		\alert{\\begin\{sideways\}}\n
		\\begin\{tabular\}\{lr\}\
		\\hline\n
		\hspace*{5ex}bianco    \& 102,5 \bs\bs\n
		\hspace*{5ex}nero      \& 15,4 \bs\bs\n
		\hspace*{10ex}\vdots\n
		\hspace*{5ex}antracite \& 15,1 \bs\bs\\hline\n
		\\end\{tabular\}\n
		\alert{\\end\{sideways\}}
	\end{LaTeXcode}
\end{frame}
% -----------------------------------------------------------
%--------------------------------------------------- SLIDE -
\begin{frame}
  \frametitle{Ruotare le tabelle}
	\begin{LaTeXoutput}
		\begin{sideways}
		\begin{tabular}{lr}\\\hline
		bianco 		& 10,2 \\
		nero   		& 15,6 \\
		giallo		& 16,6 \\
		fucsia		& 15,7 \\
		cremisi		& 12,2 \\
		amaranto	& 18,3 \\
		verde		& 11,5 \\
		grigio		& 15,3 \\
		viola		& 19,9 \\
		blu		& 14,7 \\
		rosso		& 14,4 \\
		marrone		& 17,7 \\
		rosa		& 12,9 \\
		ocra		& 19,2 \\
		arancione	& 11,8 \\
		porpora		& 14,6 \\
		celeste		& 12,9 \\
		antracite	& 15,1 \\
		\hline
		\end{tabular}
		\end{sideways}
	\end{LaTeXoutput}
  \onslide<2->
	\begin{block}{Attenzione!}
		Per ruotare degli oggetti, \`e necessario caricare il pacchetto \Lsty{rotating}
	\end{block}
\end{frame}
%-----------------------------------------------------------
%--------------------------------------------------- SLIDE -
\begin{frame}
  \frametitle{Due esempi son meglio di uno}
	\begin{center}
		\alert{\texttt{esempio\_3\_2.tex}}\\
		\alert{\texttt{esempio\_3\_3.tex}}
	\end{center}
\end{frame}
-----------------------------------------------------------
------------------------------------------------- SECTION -
\section{Formule matematiche}
%-----------------------------------------------------------
%---------------------------------------------- SUBSECTION -
\subsection{Nozioni di base}
%-----------------------------------------------------------
%--------------------------------------------------- SLIDE -
\begin{frame}
  \frametitle{A che punto siamo}
  \tableofcontents[currentsection,currentsubsection]
\end{frame}
%------------------------------------------------------------
%--------------------------------------------------- SLIDE - 
\begin{frame}
  \frametitle{L'arte della tipografia matematica}
	Generalmente la scrittura di formule matematiche costituisce la parte pi\`u complessa e delicata della stesura di un documento. Proprio in questo particolare ambito, \LaTeX\ offre una qualit\`a tipografica allo stato dell'arte.\\
  \medskip
	La sintassi per la scrittura di formule matematiche non \`e assolutamente difficile e richiede appena un minimo di pratica.
\end{frame}
%-----------------------------------------------------------
%--------------------------------------------------- SLIDE -
\begin{frame}
  \frametitle{Scrivere le formule nel testo}
	\LaTeX\ applica parecchia cura nella spaziatura nelle formule, anche quando esse sono molto semplici. Ecco un cattivo esempio di come non vanno scritte:
	\begin{LaTeXcode}
		Non \`e vero che 7+2=9 e 7-3=4, sono solo bugie.
	\end{LaTeXcode}
	\begin{LaTeXoutput}
		Non \`e vero che 7+2=9 e 7-3=4,  sono solo bugie.
	\end{LaTeXoutput}
  \smallskip
  \onslide<2->
	Il modo corretto di scrivere le formule all'interno del testo \`e quello di inserirle tra due \LCmd[]{\textdollar}\dots\LCmd[]{\textdollar}
	\begin{LaTeXcode}
		Non \`e vero che \alert{\$7+2=9\$} e \alert{\$7-3=4\$} sono solo bugie.
	\end{LaTeXcode}
	\begin{LaTeXoutput}
		Non \`e vero che $7+2=9$ e $7-3=4$, sono solo bugie.
	\end{LaTeXoutput}
\end{frame}
%-----------------------------------------------------------
%--------------------------------------------------- SLIDE -
\begin{frame}
  \frametitle{Scrivere le formule nel testo}
	Se si inserisce la formula nel testo \LaTeX\ cerca di schiacciarla per non aumentare l'interlinea.
	\begin{LaTeXcode}
		Dopo lunghi studi, Livsi\\v\{c\} \ Vr\\o dstadt riusc\`i a dimostrare che poich\'e \alert{\$\\sum\_\{i=1\}\textasciicircum n a\_i=3\$} la figura di Carducci \`e imponentemente stagliata nel panorama poetico del suo tempo.
	\end{LaTeXcode}
	\begin{LaTeXoutput}
		Dopo lunghi studi, Livsi\v{c} Vr\o dstadt riusc\`i a dimostrare che poich\'e $\sum_{i=1}^n a_i=3$ la figura di Carducci \`e imponentemente stagliata nel panorama poetico del suo tempo.
	\end{LaTeXoutput}
\end{frame}
%-----------------------------------------------------------
%--------------------------------------------------- SLIDE -
\begin{frame}
  \frametitle{Centrare le formule}
	Per centrare la formula su una riga occorre inserira tra \LCmd[]{\\[}\dots\LCmd[]{\\]} In questo caso lo sviluppo verticale sar\`a maggiore
	\begin{LaTeXcode}
		Livsi\\v\{c\} \ Vr\\o dstadt riusc\`i a dimostrare che poich\'e
		\alert{\\[\\sum\_\{i=1\}\textasciicircum n a\_i\\]} la figura di Carducci \`e imponentemente stagliata nel panorama poetico del suo tempo. 
	\end{LaTeXcode}
	\begin{LaTeXoutput}
		Livsi\v{c} Vr\o dstadt riusc\`i a dimostrare che poich\`e
			\[\sum_{i=1}^n a_i=3\]
		la figura di Carducci \`e imponentemente stagliata nel panorama poetico del suo tempo. 
	\end{LaTeXoutput}
\end{frame}
% %-----------------------------------------------------------
% %--------------------------------------------------- SLIDE -
% \begin{frame}
%   \frametitle{Centrare le formule}
% 	Per centrare la formula su una riga si usa l'ambiente \LCmd[]{displaymath} 
% 	\begin{LaTeXcode}
% 	Dimostrare che la formula:\n
% 		\alert{\\begin\{displaymath\}}\n
% 			\hspace*{5ex}\\prod\\theta\textasciicircum\{\\psi\}-45x\n
% 		\alert{\\end\{displaymath\}}\n
% 	non significa assolutamente nulla.
% 	\end{LaTeXcode}
% 	\begin{LaTeXoutput}
% 	Dimostrare che la formula:
% 		\begin{displaymath}
% 			\prod\theta^{\psi}-45x 
% 		\end{displaymath} 
% 	non significa assolutamente nulla.
% 	\end{LaTeXoutput}
% \end{frame}
%-----------------------------------------------------------
%--------------------------------------------------- SLIDE -
\begin{frame}
  \frametitle{Un esempio vale pi\`u di mille parole}
	\begin{center}
		\alert{\texttt{esempio\_3\_4.tex}}
	\end{center}
\end{frame}
%-----------------------------------------------------------
%---------------------------------------------- SUBSECTION -
\subsection{Scrivere formule matematiche}
%-----------------------------------------------------------
%--------------------------------------------------- SLIDE -
\begin{frame}
  \frametitle{A che punto siamo}
  \tableofcontents[currentsection,currentsubsection]
\end{frame}
%-----------------------------------------------------------
%--------------------------------------------------- SLIDE -
\begin{frame}
  \frametitle{Esponenti}
	Per inserire un esponente si usa il comando \LCmd[]{\textasciicircum} (accento circonflesso o \textit{circum})
 	\begin{LaTeXcode}
		\$x\alert{\textasciicircum}y\$
 	\end{LaTeXcode}
	\begin{LaTeXoutput}
		$x^y$
	\end{LaTeXoutput}
	Nel caso di esponenti pi\`u complessi si ricorre alle parentesi
 	\begin{LaTeXcode}
		\$x\textasciicircum \alert{\{y+1\}}\$
 	\end{LaTeXcode}
	\begin{LaTeXoutput}
		$x^{y+1}$
	\end{LaTeXoutput}
\end{frame}
%-----------------------------------------------------------
%--------------------------------------------------- SLIDE -
\begin{frame}
  \frametitle{Esponenti in modalit\`a testo}
	Il \LaTeX\ esiste anche la possibilit\`a di scrivere esponenti fuori dal contesto di ambienti matematici con il comando \LCmd{textsuperscript} 
 	\begin{LaTeXcode}
		Ciccio\alert{\\textsuperscript\{}\\textregistered\alert{\}}
 	\end{LaTeXcode}
	\begin{LaTeXoutput}
		Ciccio\textsuperscript{\textregistered}
	\end{LaTeXoutput}
\end{frame}
%-----------------------------------------------------------
%--------------------------------------------------- SLIDE -
\begin{frame}
  \frametitle{Indici}
	Per inserire un indice si usa il comando \LCmd[]{\textunderscore} (\textit{underscore})
 	\begin{LaTeXcode}
		\$x\alert{\textunderscore}y\$
 	\end{LaTeXcode}
	\begin{LaTeXoutput}
		$x_y$
	\end{LaTeXoutput}
	Nel caso di indici multipli si ricorre alle parentesi annidate
 	\begin{LaTeXcode}
		x\_\{n\_\{\alert{k\_\{i\}}\}\}
 	\end{LaTeXcode}
	\begin{LaTeXoutput}
		$x_{n_{k_{i}}}$
	\end{LaTeXoutput}
	I caratteri diventano via via sempre pi\`u piccoli
\end{frame}
%-----------------------------------------------------------
%--------------------------------------------------- SLIDE -
\begin{frame}
  \frametitle{Frazioni}
	Per inserire una frazione si usa il comando \LCmd{frac}
	\begin{LaTeXcode}
		\\[\n
    \hspace*{5ex}\alert{\\frac\{1\}\{a+1\}}\n
		\\]
	\end{LaTeXcode}
	\begin{LaTeXoutput}
		\[
			\frac{1}{a+1}
		\]
	\end{LaTeXoutput}
\end{frame}
%-----------------------------------------------------------
%--------------------------------------------------- SLIDE -
\begin{frame}
  \frametitle{Frazioni}
	Il comando \LCmd{frac} pu\`o anche essere annidato
	\begin{LaTeXcode}
		\\[\n
    \alert{\\frac}\{x+\alert{\\frac\{1\}\{x\}}\}\{y+\alert{\\frac\{1\}\{y\}}\}\n
		\\]
	\end{LaTeXcode}
	\begin{LaTeXoutput}
		\[
			\frac{x+\frac{1}{x}}{y+\frac{1}{y}}
		\]
	\end{LaTeXoutput}
\end{frame}
%-----------------------------------------------------------
%--------------------------------------------------- SLIDE -
\begin{frame}
  \frametitle{Radici}
	Per scrivere la radice si usa il comando \LCmd{sqrt}
 	\begin{LaTeXcode}
		\\[\n
		\hspace*{5ex}\alert{\\sqrt}[n+1]\{\bs chi + x\}\n
		\\]
 	\end{LaTeXcode}
	\begin{LaTeXoutput}
		\[
			\sqrt[n+1]{\chi + x}
		\]
	\end{LaTeXoutput}
\end{frame}
%-----------------------------------------------------------
%--------------------------------------------------- SLIDE -
\begin{frame}
  \frametitle{Sommatorie}
	Il simbolo di sommatoria si scrive con il comando \LCmd{sum}
 	\begin{LaTeXcode}
		\\[\n
		\hspace*{5ex}\alert{\\sum}\_\{i=1\}\textasciicircum \{\\infty\}\n
		\\]
 	\end{LaTeXcode}
	\begin{LaTeXoutput}
		\[
			\sum_{i=1}^{\infty}
		\]
	\end{LaTeXoutput}
\end{frame}
%-----------------------------------------------------------
%--------------------------------------------------- SLIDE -
\begin{frame}
  \frametitle{Limiti}
	I limiti si scrivono con il comando \LCmd{lim}
 	\begin{LaTeXcode}
		\\[\n
		\hspace*{5ex}\alert{\\lim}\_\{i \\to \\infty\}\n
		\\]
 	\end{LaTeXcode}
	\begin{LaTeXoutput}
		\[
			\lim_{i \to \infty}
		\]
	\end{LaTeXoutput}
\end{frame}
%-----------------------------------------------------------
%--------------------------------------------------- SLIDE -
\begin{frame}
  \frametitle{Integrali}
	Il segno di integrale si scrive con il comando \LCmd{int}\\
 	\begin{LaTeXcode}
		\\[\n
		\hspace*{5ex}\alert{\\int}\_\{a+1\}\textasciicircum \{b+1\}x\bs,dx\n
		\\]
 	\end{LaTeXcode}
	\begin{LaTeXoutput}
		\[
			\int_{a+1}^{b+1}x\,dx
		\]
	\end{LaTeXoutput}
  \onslide<2->
	\begin{block}{Attenzione!}
	il \LCmd{,} serve per inserire uno spazio prima del \textrm{\textit{dx}}
	\end{block}
\end{frame}
%-----------------------------------------------------------
%--------------------------------------------------- SLIDE -
\begin{frame}
  \frametitle{Un esempio vale pi\`u di mille parole}
	\begin{center}
		\alert{\texttt{esempio\_3\_5.tex}}
	\end{center}
\end{frame}
%-----------------------------------------------------------
%--------------------------------------------------- SLIDE -
\begin{frame}
  \frametitle{Operatori seno e coseno}
	\emph{seno} e \emph{coseno} si scrivono con i comandi \LCmd{sin} e \LCmd{cos}
	\begin{LaTeXcode}
		\\[\n
		 \hspace*{5ex}\alert{\\cos}2x=\\frac\{1-\alert{\\sin}\textasciicircum2x\}\{2\}\n
		\\]
	\end{LaTeXcode}
	\begin{LaTeXoutput}
		\[
			\cos2x=\frac{1-\sin^2x}{2}
		\]
	\end{LaTeXoutput}
  \bigskip
	\medskip Le espressioni \LCmd{sin\textasciicircum2x} e  \LCmd{sin\textasciicircum\{2\}x} sono identiche
\end{frame}
%-----------------------------------------------------------
%--------------------------------------------------- SLIDE -
\begin{frame}
  \frametitle{Operatori seno e coseno in italiano}
	Se si vuole \emph{sen x} in italiano, bisogna aggiungere nel preambolo:
	\begin{LaTeXcode}
		\\DeclareMathOperator\{\\sen\}\{sen\}
	\end{LaTeXcode}
	\medskip Poi basta scrivere \LCmd{sen\{x\}}
\end{frame}
%-----------------------------------------------------------
%--------------------------------------------------- SLIDE -
\begin{frame}
  \frametitle{Testo dentro una formula}
	Nel caso in cui occorra inserire del testo all'interno di una formula quest'ultimo deve essere dichiarato con il comando \LCmd{text}
 	\begin{LaTeXcode}
		\\[\n
		\hspace*{5ex}\\forall x\\in\\phi\alert{\\text\{\ noi abbiamo che \}}x\textasciicircum\{2\}=1\n
		\\]
 	\end{LaTeXcode}
	\begin{LaTeXoutput}
		\[
			\forall x\in\phi\text{ noi abbiamo che } x^{2}=1
		\]
	\end{LaTeXoutput}
\end{frame}
%-----------------------------------------------------------
%--------------------------------------------------- SLIDE -
\begin{frame}
  \frametitle{Un esempio vale pi\`u di mille parole}
	\begin{center}
		\alert{\texttt{esempio\_3\_6.tex}}
	\end{center}
\end{frame}
%-----------------------------------------------------------
%--------------------------------------------------- SLIDE -
\begin{frame}
  \frametitle{Parentesi a dimensione fissa}
	Posso usare parentesi di diverse dimensioni:
  \begin{columns}
  \column[t]{.4\textwidth}
	\begin{LaTeXcode}
		\LCmd[]{(} x \LCmd[]{)}\n
		\LCmd{\alert{bigl}(} x \LCmd{bigr)}\vspace{1.1ex}\n
		\LCmd{\alert{Bigl}(} x \LCmd{Bigr)}\vspace{2.2ex}\n
		\LCmd{\alert{biggr}(} x \LCmd{biggr)}\vspace{3ex}\n
		\LCmd{\alert{Biggr}(} x \LCmd{Biggr)}\n
	\end{LaTeXcode}
  \column[t]{.4\textwidth}
	\begin{LaTeXoutput}
		$( x )$\\
		$\bigl( x\bigr)$\\
		$\Bigl( x \Bigr) $\\
		$\biggr( x \biggr) $\\
		$\Biggr( x \Biggr)$
	\end{LaTeXoutput}
  \end{columns}
  \bigskip
	Da utilizzare per le formule ordinarie
\end{frame}
%-----------------------------------------------------------
%--------------------------------------------------- SLIDE -
\begin{frame}
  \frametitle{Parentesi a dimensione fissa}
	Un esempio di parentesi grande
	\begin{LaTeXcode}
		\\[\n
		 \hspace*{5ex}\alert{\\Biggl(}\\frac\{1\}\{n+1\}\alert{\\Biggr)}\textasciicircum 2\n
		\\]
	\end{LaTeXcode}
	\begin{LaTeXoutput}
		\[
			\Biggl(\frac{1}{n+1}\Biggr)^2
		\]
	\end{LaTeXoutput}
	Ovviamente \LCmd{Bigl} accetta anche parentesi quadre e graffe
\end{frame}
%-----------------------------------------------------------
%--------------------------------------------------- SLIDE -
\begin{frame}
  \frametitle{Parentesi automatiche}
	Per ottenere delle parentesi che si adattano alle dimensioni di quello che contengono si usa \LCmd{left(} e \LCmd{right)} e analogamente per quadre e graffe.
	\begin{block}{Attenzione}
		Le graffe sono un carattere riservato quindi si scrive \LCmd{left\bs\lb} e \LCmd{right\bs\rb}
	\end{block}
  \bigskip
	Da utilizzare per elementi di `grosse' dimensioni quando non se ne conosce la dimensione (matrici, casi, etc)
\end{frame}
%-----------------------------------------------------------
%--------------------------------------------------- SLIDE -
\begin{frame}
  \frametitle{Parentesi automatiche}
	Un esempio di parentesi grande
	\begin{LaTeXcode}
		\\[\n
		 \hspace*{5ex}\alert{\\left(}\\frac\{1\}\{n+1\}\alert{\\right)}\textasciicircum 2\n
		\\]
	\end{LaTeXcode}
	\begin{LaTeXoutput}
		\[
			\left(\frac{1}{n+1}\right)^2
		\]
	\end{LaTeXoutput}
	Ovviamente \LCmd{Bigl} accetta anche parentesi quadre e graffe
\end{frame}
%-----------------------------------------------------------
%--------------------------------------------------- SLIDE -
\begin{frame}
  \frametitle{Alcune lettere greche}
	Scrivere lettere greche all'interno di ambienti matematici \`e estremamente semplice
	\begin{columns}
	  \column[t]{.3\textwidth}
		\begin{LaTeXcode}
			\\alpha\n
			\\beta
		\end{LaTeXcode}
		\begin{LaTeXcode}
			\\xi\n
			\\Xi\n
			\\gamma\n
			\\Gamma\n
			\\omega\n
			\\Omega
		\end{LaTeXcode}
	  \column[t]{.3\textwidth}
		\begin{LaTeXoutput}
			$\alpha$\\
			$\beta$
		\end{LaTeXoutput}
		\begin{LaTeXoutput}
			$\xi$\\
			$\Xi$\\
			$\gamma$\\
			$\Gamma$\\
			$\omega$\\
			$\Omega$
		\end{LaTeXoutput}
	\end{columns}
\end{frame}
%-----------------------------------------------------------
%--------------------------------------------------- SLIDE -
\begin{frame}
  \frametitle{Simboli matematici}
	\LaTeX\ mette a disposizione una collezione pressoch\'e completa di simboli matematici. Questi di seguito costituiscono solo un esempio minimale.
  \begin{columns}
  \column[t]{.4\textwidth}
	\begin{LaTeXcode}
		\$\\leftarrow\$\n
		\$\\curvearrowleft\$\n
		\$\\looparrowleft\$\n
		\$\\precsim\$\n
		\$\\gnapprox\$
	\end{LaTeXcode}
  \column[t]{.4\textwidth}
	\begin{LaTeXoutput}
		$\leftarrow$\\[.1ex]
		$\curvearrowleft$\\[.1ex]
		$\looparrowleft$\\[.1ex]
		$\precsim$\\[.1ex]
		$\gnapprox$
	\end{LaTeXoutput}
  \end{columns}
	\begin{block}{Attenzione!}
	Per utilizzare i simboli matematici pi\`u comuni, \`e necessario caricare il pacchetto \Lsty{amssymb}
	\end{block}
\end{frame}
%-----------------------------------------------------------
%--------------------------------------------------- SLIDE -
\begin{frame}
  \frametitle{Un esempio vale pi\`u di mille parole}
	\begin{center}
		\alert{\texttt{esempio\_3\_7.tex}}
	\end{center}
\end{frame}
%-----------------------------------------------------------
%---------------------------------------------- SUBSECTION -
\subsection{Ambienti matematici}
%-----------------------------------------------------------
%--------------------------------------------------- SLIDE -
\begin{frame}
  \frametitle{A che punto siamo}
  \tableofcontents[currentsection,currentsubsection]
\end{frame}
%-----------------------------------------------------------
%--------------------------------------------------- SLIDE -
\begin{frame}
  \frametitle{Scrivere le equazioni}
	L'ambiente \LCmd[]{equation} permette di numerare le equazioni
	\begin{LaTeXcode}
		\alert{\\begin\{equation\}}\n
		\hspace*{5ex} F(x):= \\int\_a\textasciicircum x f(x)\bs,dx,\n
		\alert{\\end\{equation\}}
	\end{LaTeXcode}
	\begin{LaTeXoutput}
		\begin{equation}
			F(x):= \int_a^x f(x)\,dx,
		\end{equation}
	\end{LaTeXoutput}
	\begin{block}{Attenzione!}
		Per utilizzare questo ambiente \`e necessario caricare il pacchetto \Lsty{amsmath}
	\end{block}
\end{frame}
%-----------------------------------------------------------
%--------------------------------------------------- SLIDE -
\begin{frame}
  \frametitle{Scrivere le equazioni}
	Con il simbolo \LCmd[]{\textsuperscript{*}} le equazioni non vengono pi\`u numerate
	\begin{LaTeXcode}
		\\begin\{equation\alert{\textsuperscript{*}}\}\n
		\hspace*{5ex} F(x):= \\int\_a\textasciicircum x f(x)\bs,dx,\n
		\\end\{equation\alert{\textsuperscript{*}}\}
	\end{LaTeXcode}
	\begin{LaTeXoutput}
		\begin{equation*}
			F(x):= \int_a^x f(x)\,dx,
		\end{equation*}
	\end{LaTeXoutput}
\end{frame}
%-----------------------------------------------------------
%--------------------------------------------------- SLIDE -
\begin{frame}
  \frametitle{Gli ambienti per le matrici}
	Matrici	senza parentesi
	\begin{LaTeXcode}<2->
		matrix
	\end{LaTeXcode}
	Matrici con parentesi tonde (con delimitatori \LCmd[]{(}\;\LCmd[]{)})
	\begin{LaTeXcode}<3->
		pmatrix
	\end{LaTeXcode}
	Matrici con parentesi quadre (con delimitatori \LCmd[]{\ls\;\rs})
	\begin{LaTeXcode}<4->
		bmatrix
	\end{LaTeXcode}
	Matrici con parentesi graffe (con delimitatori \LCmd[]{\tlb}\;\LCmd[]{\trb})
	\begin{LaTeXcode}<5->
		Bmatrix
	\end{LaTeXcode}
\end{frame}
%-----------------------------------------------------------
%--------------------------------------------------- SLIDE -
\begin{frame}
  \frametitle{Gli ambienti per le matrici}
	Matrici con barre verticali (con delimitatori \LCmd[]{|}\;\LCmd[]{|})
	\begin{LaTeXcode}<2->
		vmatrix
	\end{LaTeXcode}
	Matrici con doppie barre verticali (con delimitatori \LCmd[]{||}\;\LCmd[]{||})
	\begin{LaTeXcode}<3->
		Vmatrix
	\end{LaTeXcode}
	Matrici di piccola dimensione (per essere facilmente inserite nel testo)
	\begin{LaTeXcode}<4->
		smallmatrix
	\end{LaTeXcode}
\end{frame}
%-----------------------------------------------------------
%--------------------------------------------------- SLIDE -
\begin{frame}
  \frametitle{Scrivere matrici senza parentesi}
	\begin{LaTeXcode}
		\\[\n
		\alert{\\begin\{matrix\}}\n
		\hspace*{5ex} 1-x \& 2 \bs\bs\n
		\hspace*{5ex} 3   \& 4-x \n
		\alert{\\end\{matrix\}}\n
		\\]
	\end{LaTeXcode}
	\begin{LaTeXoutput}
		\[
		\begin{matrix}
		1-x & 2 \\
		3 & 4-x
		\end{matrix}
		\]
	\end{LaTeXoutput}
\end{frame}
%-----------------------------------------------------------
%--------------------------------------------------- SLIDE -
\begin{frame}
  \frametitle{Esempio: matrice con parentesi tonde e puntini}
	\begin{LaTeXcode}
		\\[\n
		\alert{\\begin\{pmatrix\}}\n
		\hspace*{5ex} a\_\{11\}\ \& a\_\{12\}\ \& \\dots \& a\_\{1n\}\bs\bs\n
		\hspace*{5ex} a\_\{21\}\ \& a\_\{22\}\ \& \alert{\\dots} \& a\_\{2n\} \bs\bs\n
		\hspace*{5ex} \\vdots \& \\vdots \& \alert{\\ddots} \& \alert{\\vdots} \bs\bs\n
		\hspace*{5ex} a\_\{n1\}\ \& a\_\{n2\}\ \& \\dots \& a\_\{nn\}\bs\bs\n
		\alert{\\end\{pmatrix\}}\n
		\\]
	\end{LaTeXcode}
	\begin{LaTeXoutput}
		\[
		\begin{pmatrix}
		a_{11} & a_{12} & \dots & a_{1n} \\
		a_{21} & a_{22} & \dots & a_{2n} \\
		\vdots & \vdots & \ddots & \vdots \\
		a_{n1} & a_{n2} & \dots & a_{nn}
		\end{pmatrix}
		\]
	\end{LaTeXoutput}
\end{frame}
%-----------------------------------------------------------
%--------------------------------------------------- SLIDE -
\begin{frame}
  \frametitle{Un esempio vale pi\`u di mille parole}
	\begin{center}
		\alert{\texttt{esempio\_3\_8.tex}}
	\end{center}
\end{frame}
%-----------------------------------------------------------
%--------------------------------------------------- SLIDE -
\begin{frame}
  \frametitle{L'ambiente \texttt{array}}
	Viene utilizzato per scrivere sistemi di equazioni
  \begin{columns}
  \column[t]{.5\textwidth}
	\begin{LaTeXcode}
		\\[\n
		\alert{\\begin\{array\}\{l\}} \n
		\hspace*{5ex} x+y+z=0\bs\bs \n
		\hspace*{5ex} 2x-y=1\bs\bs \n
		\hspace*{5ex} y-4z=-3 \n
		\alert{\\end\{array\}} \n
		\\end\{displaymath\} 
	\end{LaTeXcode}
  \column[t]{.5\textwidth}
	\begin{LaTeXoutput}
		\vspace{4.6ex}
		\begin{displaymath} 
		\begin{array}{l}
		x+y+z=0\\
		2x-y=1\\
		y-4z=-3 
		\end{array} 
		\end{displaymath} 
		\vspace{4.6ex}
	\end{LaTeXoutput}
  \end{columns}
\end{frame}
%-----------------------------------------------------------
%--------------------------------------------------- SLIDE -
\begin{frame}
  \frametitle{L'ambiente \texttt{array}}
	Il comando \LCmd{left\\\{} Aggiunge una graffa alla sola sinistra
  \begin{columns}
  \column[t]{.5\textwidth}
	\begin{LaTeXcode}
		\\[\n
		\alert{\\left\bs\{} \n
		\\begin\{array\}\{l\} \n
		\hspace*{5ex} x+y+z=0\bs\bs \n
		\hspace*{5ex} 2x-y=1\bs\bs \n
		\hspace*{5ex} y-4z=-3 \n
		\\end\{array\} \n
		\alert{\\right.} \n
		\\]
	\end{LaTeXcode}
  \column[t]{.5\textwidth}
	\begin{LaTeXoutput}
		\vspace{5.6ex}
		\[
		\left\{
		\begin{array}{l}
		x+y+z=0\\
		2x-y=1\\
		y-4z=-3 
		\end{array} 
		\right. 
		\] 
		\vspace{5.6ex}
	\end{LaTeXoutput}
  \end{columns}
\end{frame}
%-----------------------------------------------------------
%--------------------------------------------------- SLIDE -
\begin{frame}
  \frametitle{L'ambiente \texttt{cases}}
	Viene utilizzato per scrivere definizioni costituite per casi
	\begin{LaTeXcode}
		\\[\n
		f(n):=\n
		\alert{\\begin\{cases\}} \n
		\hspace*{5ex} 2n+1 \& \\text\{se \$n\$ \`e dispari,\}\bs\bs\n
		\hspace*{5ex} n/2  \& \\text\{se \$n\$ \`e pari.\}\bs\bs\n
		\alert{\\end\{cases\}} \n
		\\]
	\end{LaTeXcode}
	\begin{LaTeXoutput}
		\[
		f(n):=
		\begin{cases} 
		2n+1 & \text{se $n$ \`e dispari,}\\ 
		n/2  & \text{se $n$ \`e pari.} \\
		\end{cases}
		\]
	\end{LaTeXoutput}
\end{frame}
%-----------------------------------------------------------
%--------------------------------------------------- SLIDE -
\begin{frame}
  \frametitle{L'ambiente \texttt{multline}}
	Viene utilizzato per scrivere per un'equazione da dividere in pi\`u righe, senza particolari allineamenti
	\begin{LaTeXcode}
		\alert{\\begin\{multline\}}\n
		\hspace*{5ex} f=a+b+c+d+e+g+h \bs\bs\n
		\hspace*{5ex} +i+k+l+m+n+o+\bs\bs\n
		\hspace*{5ex} +p+q+r+s+t+u+v \n
		\alert{\\end\{multline\}} 
	\end{LaTeXcode}
	\begin{LaTeXoutput}
		\begin{multline} 
		f=a+b+c+d+e+g+h \\ 
		+i+k+l+m+n+o+\\ 
		+p+q+r+s+t+u+v 
		\end{multline} 
	\end{LaTeXoutput}
\end{frame}
%-----------------------------------------------------------
%--------------------------------------------------- SLIDE -
\begin{frame}
  \frametitle{L'ambiente \texttt{CD}}
	Viene utilizzato per scrivere diagrammi commutativi
  \begin{columns}
  \column[t]{.5\textwidth}
	\begin{LaTeXcode}
		\\[\n
		\alert{\\begin\{CD\}}\n
		\hspace*{5ex} A	@>a>b> B \bs\bs\n
		\hspace*{5ex} @VcVV	@AAdA\bs\bs\n
		\hspace*{5ex} C 	@= D\n
		\alert{\\end\{CD\}}\n
		\\]
	\end{LaTeXcode}
  \column[t]{.5\textwidth}
	\begin{LaTeXoutput}
		\vspace{3ex}
		\[
		\begin{CD}
		A	@>a>b> B \\
		@VcVV	@AAdA\\
		C	@= D\\
		\end{CD}
		\]
		\vspace{3ex}
	\end{LaTeXoutput}
  \end{columns}
	\begin{block}{Attenzione!}
	\`E necessario caricare il pacchetto \Lsty{amscd}
	\end{block}
\end{frame}
-----------------------------------------------------------
--------------------------------------------------- SLIDE -
\begin{frame}
  \frametitle{Un esempio vale pi\`u di mille parole}
	\begin{center}
		\alert{\texttt{esempio\_3\_9.tex}}
	\end{center}
\end{frame}
-----------------------------------------------------------
--------------------------------------------------- SLIDE -
\begin{frame}
  \frametitle{Le possibilit\`a sono molto vaste}
	\tiny\vspace{-.08\textheight}
	\[\xymatrix{
	(M,h,z) \ar[dd]^{\pi_0} \ar[dr]^\alpha_\cong \ar[rr]^{\pi_1}
	&& (M_1,h_1,0) \ar'[d]^-{\pi_{1d}}[dd] \ar[dr]^{\alpha_1}_\cong
	\\
	& (M',h',z')\oplus H(\Lambda^k) \ar[dd]^<(.25){\pi_0} \ar[rr]^<(.25){\pi_1}
	&& (M'_1,h'_1,0)\oplus H(\Lambda_1^k) \ar[dd]^{\pi_{1d}}
	\\
	(M_0,h_0,z_0) \ar@{=}[dd] \ar[dr]^{\alpha_0}_\cong \ar'[r]^<(.6){\pi_{0d}}[rr]
	&& (M_d,h_d,0) \ar@{=}'[d][dd] \ar[dr]^{\alpha_d}_\cong
	\\
	& (M'_0,h'_0,z'_0)\oplus H(\Lambda_0^k) \ar[dd]^<(.25){\beta'_0\oplus\text{id}}_<(.25)\cong
	\ar[rr]^<(.25){\pi_{0d}}
	&& (M'_d,h'_d,0)\oplus H(\Lambda_d^k) \ar[dd]^{\beta'_d\oplus\text{id}}_\cong
	\\
	(M_0,h_0,z_0) \ar[dr]^{\beta_0}_\cong \ar'[r]^<(.6){\pi_{0d}}[rr]
	&& (M_d,h_d,0) \ar[dr]^{\beta_d}_\cong
	\\
	& (L,\lambda,x)\oplus H(\Lambda_0^k) \ar[rr]^{\pi_{0d}}
	&& (L_d,\lambda_d,0)\oplus H(\Lambda_d^k)
	}\]
\end{frame}
%-----------------------------------------------------------
%------------------------------------------------- SECTION -
\section{Riferimenti bibliografici}
%-----------------------------------------------------------
%--------------------------------------------------- SLIDE -
\begin{frame}
  \frametitle{Abbiamo quasi finito}
  \tableofcontents[currentsection,currentsubsection]
\end{frame}
%-----------------------------------------------------------
%--------------------------------------------------- SLIDE -
\begin{frame}
  \frametitle{Realizzare la bibliografia}
	La bibliogafia \`e una lista di pubblicazioni generalmente usati per preparare un documento. \\
  \medskip
	\LaTeX\ permette di gestire con grandissima efficienza sia la bibliografia che i riferimenti ad essa presenti nel testo, il tutto in accordo con diverse regole bibliografiche adottate.\\
  \bigskip
  \onslide<2->
	In \LaTeX\ esistono due metodi per realizzare la bibliografia.
\end{frame}
%-----------------------------------------------------------
%--------------------------------------------------- SLIDE -
\begin{frame}
  \frametitle{L'ambiente \texttt{thebibliograpy}}
	\begin{LaTeXcode}
		\alert{\\begin\{thebibliograpy\}\{666\}}\n
		\hspace*{5ex}\\bibitem\{VBF\}\n
		\hspace*{5ex}A.\textasciitilde Vitali, L.\textasciitilde Buzzanca, P.\textasciitilde Franco,\n
		\hspace*{5ex}\\textit\{Fisica dei Quanti\},\n
		\hspace*{5ex}un approccio basato sulla teoria di Kirchhoff\n
		\alert{\\end\{thebibliograpy\}}
	\end{LaTeXcode}
	\begin{LaTeXoutput}
		\textbf{\Large Bibliografia}\\[2ex]
		[1] A.~Vitali, L.~Buzzanca, P.~Franco, \textit{Fisica dei Quanti}, un approccio basato sulla teoria di Kirchhoff
	\end{LaTeXoutput}
\end{frame}
%-----------------------------------------------------------
%--------------------------------------------------- SLIDE -
\begin{frame}
  \frametitle{Citazioni}
	\begin{LaTeXcode}
		Oggi nel 2010, il nostro paese \`e invidiato e temuto, anche se \`e tuttora accerchiato dai centri sociali, dall'Europa bolscevica e dai molli americani del primo presidente ex nero Michael Jackson \alert{\\cite\{VBF\}}.
	\end{LaTeXcode}
	\begin{LaTeXoutput}
		Oggi nel 2010, il nostro paese \`e invidiato e temuto, anche se \`e tuttora accerchiato dai centri sociali, dall'Europa bolscevica e dai molli americani del primo presidente ex nero Michael Jackson [1].
	\end{LaTeXoutput}
\end{frame}
%-----------------------------------------------------------
%--------------------------------------------------- SLIDE -
\begin{frame}
  \frametitle{Un esempio vale pi\`u di mille citazioni}
	\begin{center}
		\alert{\texttt{esempio\_3\_10.tex}}
	\end{center}
\end{frame}
%-----------------------------------------------------------
%--------------------------------------------------- SLIDE -
\begin{frame}
  \frametitle{Bib\TeX}
	Bib\TeX\ permette di gestire in modo flessibile ed automatico le citazioni bibliografiche. 
	Per far questo si ricorre a due file supplementari
  \bigskip
	\begin{itemize}
		\item \Lsty{bibliografia.bib} per le referenze bibliografiche
		\item \Lsty{stile.bst} per lo stile bibliografico adottato
	\end{itemize}
\end{frame}
%-----------------------------------------------------------
%--------------------------------------------------- SLIDE -
\begin{frame}
  \frametitle{Il file \texttt{bibliografia.bib}}
	\begin{LaTeXcode}
		@book\{\alert{VBF},\n
		\hspace*{5ex} author=\{A.\textasciitilde Vitali, L.\textasciitilde Buzzanca, P.\textasciitilde Franco\},\n
		\hspace*{5ex} title=\{Fisica dei Quanti\},\n
		\hspace*{5ex} publisher = \{Astrophysical Journal\},\n
		\hspace*{5ex} year = \{2010\},\n
		\}
	\end{LaTeXcode}
\end{frame}
%-----------------------------------------------------------
%--------------------------------------------------- SLIDE -
\begin{frame}
  \frametitle{Il file \texttt{bibliografia.bib}}
	Oltre a \LCmd[]{@book}, Bib\TeX\ permette l'uso diversi altri tipi di \textit{entries}:
	\begin{itemize}
		\item\Lopt{article}
		\item\Lopt{booklet}
		\item\Lopt{inbook}
		\item\Lopt{incollection}
		\item\Lopt{inproceedings}
		\item\Lopt{manual}
		\item\dots
		\item\Lopt{unpublished}
		\item\Lopt{misc}
	\end{itemize}
\end{frame}
%-----------------------------------------------------------
%--------------------------------------------------- SLIDE -
\begin{frame}
  \frametitle{Stili bibliografici}
	Vediamo come procedere.\\
  \medskip
	Nel testo del documento posso inserire la bibliografia richiamando il file con le referenze e lo stile bibliografico prescelto con i comandi:
	\begin{LaTeXcode}
		\\bibliography\{bibliografia\}\n
		\\bibliographystyle\{plain\}
	\end{LaTeXcode}
\end{frame}
%-----------------------------------------------------------
%--------------------------------------------------- SLIDE -
\begin{frame}
  \frametitle{Il comando \LCmd{cite} con plain}
	Questo stile si caratterizza per le citazioni numeriche racchiuse tra parentesi quadre ed i riferimenti ordinati alfabeticamente
	\begin{columns}
	  \column[t]{.45\textwidth}
		\begin{LaTeXcode}
		\\cite\{wr\}\n 	
		\\cite\ls e.g.\rs\ls\rs\{wr\}\n  
		\\cite\ls pg.\textasciitilde2\rs\{wr\}\n 
		\\cite\{wr,fs\}\n 	
		\\nocite\{wr,fs\}
		\end{LaTeXcode}
	  \column[t]{.55\textwidth}
		\begin{LaTeXoutput} % workaround maledettamente sporco
		\char'133 1\char'135\\
		\char'133 e.g. 1\char'135\\
		\char'133 1, pg.\ 2\char'135\\
		\char'133 1,2\char'135\\
		\ \\
		\end{LaTeXoutput}
	\end{columns}
\end{frame}
%-----------------------------------------------------------
%--------------------------------------------------- SLIDE -
\begin{frame}
  \frametitle{Il comando \LCmd{cite} con natbib}
	Questo stile si caratterizza per le citazioni estese racchiuse tra parentesi tonde ed i riferimenti ordinati alfabeticamente %verificare
	\begin{columns}
	  \column[t]{.45\textwidth}
		\begin{LaTeXcode}
		\\citet\{wr\}\n  
		\\citep\{wr\}\n 
		\\citet\ls pg.\textasciitilde2\rs\{wr\}\n 
		\\citep\ls e.g.\rs\ls\rs\n 
		\\citep\{wr,fs\}\n 
		\\nocite\{wr\}
		\end{LaTeXcode}
	  \column[t]{.55\textwidth}
		\begin{LaTeXoutput} % workaround maledettamente sporco
		Wrodstat (2000)\\
		(Wrodstat, 2000)\\
		Wrodstat (2000, pg. 2)\\
		(e.g. Wrodstat, 2000)\\
		(Wrodstat, 2000; Fish, 1999)\\
		\ \\
		\end{LaTeXoutput}
	\end{columns}
  \onslide<2->
	\begin{block}{Attenzione!}
	In questo caso \`e necessario caricare anche il pacchetto \Lsty{natbib}
	\end{block}
\end{frame}
%-----------------------------------------------------------
%--------------------------------------------------- SLIDE -
\begin{frame}
  \frametitle{Altri stili bibliografici}
	Citazioni alfanumeriche con le iniziali dell'autore e l'anno di pubblicazione
	\begin{LaTeXcode}<2->
		alpha
	\end{LaTeXcode}
	Citazioni numeriche e riferimenti ordinati nell'ordine in cui si trovano
	\begin{LaTeXcode}<3->
		unstr
	\end{LaTeXcode}
	Citazioni in \textsc{Small Caps}
	\begin{LaTeXcode}<4->
		acm
	\end{LaTeXcode}
\end{frame}
%-----------------------------------------------------------
%--------------------------------------------------- SLIDE -
\begin{frame}
  \frametitle{\dots e molti altri ancora}
	\vspace*{-2ex}\begin{columns}
	  \column[t]{.21\textwidth}
		\begin{LaTeXoutput}
			\Lopt{ieeetr}\\
			\Lopt{unsrt}\\
			\Lopt{IEEE}\\
			\Lopt{ama}\\
			\Lopt{cj}\\
			\Lopt{nar}\\
			\Lopt{nature}\\
			\Lopt{phjcp}\\
			\Lopt{is-unsrt}\\	
			\Lopt{plain}\\
			\Lopt{abbrv}
		\end{LaTeXoutput}
	  \column[t]{.25\textwidth}
		\begin{LaTeXoutput}
			\Lopt{acm}\\
			\Lopt{siam}\\
			\Lopt{jbact}\\
			\Lopt{amsplain}\\
			\Lopt{finplain}\\
			\Lopt{IEEEannot}\\
			\Lopt{is-abbrv }\\	
			\Lopt{is-plain}\\
			\Lopt{annotation}\\
			\Lopt{plainyr}\\
			\Lopt{decsci}
		\end{LaTeXoutput}
	  \column[t]{.21\textwidth}
		\begin{LaTeXoutput}
			\Lopt{jtbnew}\\
			\Lopt{neuron}\\
			\Lopt{cell}\\
			\Lopt{jas99}\\
			\Lopt{abbrvnat}\\	
			\Lopt{ametsoc}\\
			\Lopt{apalike}\\
			\Lopt{jqt1999}\\
			\Lopt{plainnat}\\
			\Lopt{jtb}\\
			\Lopt{humanbio}
		\end{LaTeXoutput}
	  \column[t]{.27\textwidth}
		\begin{LaTeXoutput}
			\Lopt{these}\\
			\Lopt{chicagoa}\\
			\Lopt{development }\\	
			\Lopt{unsrtnat}\\
			\Lopt{amsalpha}\\
			\Lopt{alpha}\\
			\Lopt{annotate}\\
			\Lopt{is-alpha}\\
			\Lopt{wmaainf}\\
			\Lopt{alphanum}\\
			\Lopt{apasoft} 
		\end{LaTeXoutput}
	\end{columns}
\end{frame}
%-----------------------------------------------------------
%--------------------------------------------------- SLIDE -
\begin{frame}
  \frametitle{Un esempio vale pi\`u di mille citazioni}
	\begin{center}
		\alert{\texttt{esempio\_3\_11.tex}}
	\end{center}
\end{frame}
%-----------------------------------------------------------
%--------------------------------------------------- SLIDE -
\begin{frame}
  \frametitle{Per oggi abbiamo finito}
	\begin{center}
		\huge
		Grazie e alla prossima lezione
	\end{center}
  \medskip
	\begin{block}{Cosa impareremo la prossima volta}
		\begin{itemize}
			\item inserire e disegnare le \textbf{figure}
			\item fare delle bellissime \textbf{presentazioni a video}
			\item iniziamo a lavorare sulla tesi di laurea 
			\item come \textbf{tirarvi fuori dai guai} con le vostre gambe
		\end{itemize}
	\end{block}
\end{frame}
%-----------------------------------------------------------
%----------------------------------------------------- END -